\begin{abstract}

Este trabajo presenta la aplicaci�n de t�cnicas de computaci�n paralela usando una GPU (Graphics Processing Unit) para mejorar la eficiencia computacional del algoritmo de \emph{ray tracing}.

El trabajo ofrece un relevamiento del estado del arte de los algoritmos de generaci�n de im�genes foto-realistas por computadora, haciendo �nfasis en el algoritmo de \emph{ray tracing}. Asimismo, se presenta un estudio general sobre la utilizaci�n de GPUs como plataforma de ejecuci�n de aplicaciones paralelas, y un relevamiento de las estrategias de aceleraci�n del algoritmo de \emph{ray tracing} y sus implementaciones sobre dicha plataforma. Por �ltimo, se detalla un relevamiento del estado del arte de algoritmos de \emph{ray tracing} que alcanzan tiempos de generaci�n de imagen que permiten interactividad. 

Se propone el dise�o e implementaci�n de tres versiones del algoritmo de \emph{ray tracing} sobre GPU, donde cada versi�n implementada se corresponde con un hito del proceso de desarrollo. Los tres hitos alcanzados en orden cronol�gico son: implementaci�n completa del algoritmo de \emph{ray tracing} sobre GPU, introducci�n de estrategias que permiten explotar los distintos niveles de memoria provistos por las GPUs e introducci�n de optimizaciones al proceso de intersecci�n entre los rayos y la escena. Adem�s, cada versi�n para GPU tiene su versi�n equivalente implementada para CPU, con el prop�sito de evaluar el desempe�o computacional entre ellas.

Se detalla el dise�o y construcci�n de un \emph{benchmark} propio para la e\-va\-lua\-ci�n de las distintas versiones implementadas. Los principales lineamientos que marcan el dise�o del mismo son: explotar puntos d�biles de las versiones implementadas, estudiar el comportamiento del algoritmo frente a variaciones de la cantidad de primitivas de las escenas y comparar con algoritmos similares implementados en otros proyectos.

Los resultados experimentales obtenidos demuestran que existe una reducci�n del tiempo de ejecuci�n significativa frente al algoritmo implementado en CPU, marcando un importante avance hacia la generaci�n de im�genes en tiempo real sobre computadoras de escritorio. En cuanto a la comparaci�n con los algoritmos estado del arte en \emph{ray tracing} sobre GPU se lograron resultados comparables.
\\
\Keywords{computaci�n gr�fica, GPU, GPGPU, paralelismo, \emph{ray tracing}.}
\end{abstract}



