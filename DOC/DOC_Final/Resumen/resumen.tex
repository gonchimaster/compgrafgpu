\begin{abstract}

Este trabajo presenta la aplicaci�n de t�cnicas de computaci�n paralela usando una GPU (Graphic Processing Unit) para mejorar la eficiencia computacional del algoritmo de \emph{ray tracing}.

El trabajo presenta un relevamiento del estado del arte de los algoritmos de generaci�n de im�genes foto-realistas por computadora, se estudian las diferentes alternativas disponibles y las posibilidades de paralelizaci�n de cada una. Se investigan las paralelizaciones existentes del algoritmo de \emph{ray tracing} tanto en clusters como en GPU. As� mismo se realiza un relevamiento de las diferentes estrategias para verificar la calidad de las im�genes generadas y se selecciona un conjunto de casos de prueba relevantes.

Se obtiene el dise�o e implementaci�n de tres versiones del algoritmo de \emph{ray tracing} con una performance comparable a las mejores soluciones que ejecutan en GPU, as� como una implementaci�n en CPU para realizar la comparaci�n de performance. Para lograr esta performance se tuvo que evaluar alternativas de optimizaci�n y la utilizaci�n de estructuras de aceleraci�n. Se eligen los casos de prueba y realiza una evaluaci�n de los resultados obtenidos en las mismas.

Los resultados experimentales obtenidos a partir de las implementaciones propuestas demuestran que existe una significante reducci�n del tiempo de ejecuci�n frente a implementaciones para CPU, marcando un importante avance hacia la generaci�n de im�genes en tiempo real sobre computadoras de escritorio.
\\
\\
\Keywords{computaci�n gr�fica, GPU, GPGPU, paralelismo, \emph{ray tracing}.}
\end{abstract}



